\chapter{Contributions and perspectives} \label{Contributions and perspectives}
\minitoc

\section{Contributions}

In this chapter we review the contributions of this thesis, and discuss possible tracks for future research.

\section{Perspectives}

\subsection{Process improvement perspectives}


\subsection{Quality control improvement perspectives}

\subsection{The Digital Twin}
* Simulation data allows to start working on a new production before the first parts are blown.

* Simulation data can be coupled with experimental data

\subsection{Add the tank effect}

In chapter 4 we have presented our solution to infer the thickness of a blow-molded part by exploiting the surface decay temperature of the same part right after the part has been blown. The use of a thermal camera allows us to take into account, through the acquisition of a sequence of frames FINIRE.
Through the use of Machine Learning and Deep Learning we have proven to be able to model the relationship between the temperature evolution over time of some critical point and the corresponding thickness. First results proves the interest of our solution, but they highlight a drawback: the model is not able to generalize on FINIRE. One possible solution for imporving the ability of the model to predcit unseen points would be to combine the Temporal and the Spatial-Temporal approach. 

\subsection{How to handle the model reliability}

In section REF we have seen how one of