\chapter*{Introduction}
\addcontentsline{toc}{chapter}{Introduction}
\thispagestyle{empty}


The development of new technologies such as Machine Learning (and Deep Learning), IoT and Cloud Computing are opening up new research perspectives in the manufacturing industry. Industry 4.0 holds the promise of increased flexibility in manufacturing, along with mass customisation, better quality, and improved productivity. In this context, Plastic Omnium Clean Energy System aims to leverage these new technologies in order to keep its leadership in the manufacturing industry of fuel tanks. For Plastic Omnium Clean Energy System, Industry 4.0 is a new way of looking at performance, with a more precise and immediate vision (based on real-time indicators) of the entire production chain, but also the optimisation of production through the use of data-driven methods. In the context of an interconnected plant, the large amount of data collected from different sources—production equipment and systems as well as enterprise—can be helpful in taking decision and contributes to a continuous improvement process. In particular, we think that the integration of machine learning models inside a complex industrial process can reduce the non-quality costs with the increase of the Overall Equipment Effectiveness (OEE). This research work will focus on the quality improvement of the fuel tanks produced through the extrusion blow-moulding manufacturing process. Extrusion blow-moulding process takes a thin-walled tube called a \textit{parison} that has been formed by extrusion, entraps it between two halves of a larger diameter mould, and then expands it by blowing air into the tube, forcing the parison out against the moulds. The fuel tank produced through this manufacturing process must respect some dimensional and geometrical constraints to comply with the customer specifications. The thickness of the tank over the whole surface must be sufficient to ensure the robustness of the part and therefore its safety, while avoiding an excessive and unnecessary weight of the finished product. Unfortunately, measuring the thickness of a hollow part is a time-consuming operation that requires several minutes of work and that cannot be done online for each part. As a consequence, only a subset of the produced parts can be measured. One set of statistical tools for applying such a screening is acceptance sampling. Using such tools enables decision makers to determine what action to take on a batch of products. Decisions based on frequency testing, rather than on 100\% inspection, are more expedient and cost effective but it cannot guarantee the conformity of all parts of the population from which the sample was drawn. In order to reinforce the control of the parts, the tank weight is measured for 100\% of the manufactured parts. The weight is an indicator of how much material is composing the part and allows for overall control of the quality of the part. Unlike the thickness, which has to be measured in several areas of the tank and cannot be carried out online for all the parts, the weight can be easily measured for all part and can provide an overall information about the amount of material composing the fuel tank. This thesis explores how data-driven methods and, in particular, machine learning and deep learning could be applied in the industrial context of the extrusion blow-moulding in order to improve the quality of the produced fuel tanks. Supervised machine learning is proposed as a tool to discover hidden patterns between the process parameters of the machine and the quality of the parts that have been manufactured.

In our opinion, the overall quality improvement of the manufactured parts could be achieved in two ways:

\begin{itemize}
    \item Through the manufacturing process optimisation.   
    \item By improving the quality control. By enhancing the quality control through a 100\% inspection of the part it is possible to react faster to quality non-conformities and to avoid to send to the customers some parts which do not comply to the specifications which may cause some Quality Recall.  
\end{itemize}
%
We claim that Machine Learning, and more in general data-driven methods, could be either used to optimise the process and the quality control. By modelling the relationship between the process and the quality data, using a data-driven method, it is possible to infer the quality of a part given a new set of input data. Moreover, by leveraging interpretable Machine Learning algorithms it is eventually possible to identify which parameters affect most the quality of the final part.    

The experimental part will be predominant in this research work. Firstly, we rely on experimentation and measurement to get all the data needed to build the statistical algorithms. The machine will be equipped with new sensors, such as \textit{RGB} or thermal cameras in order to collect a new set of previously unexplored data. These new sources of data, combined with the process data already available within the company, will constitutes the entry point for training our machine learning models. In addition, this research work has made it possible to work on a few industrial software-based applications which bring value to the overall extrusion blow-moulding manufacturing process. The development of these applications is an outcome of our work on data analysis and it constitutes a part of the contributions of this research work.

\section*{Thesis structure}

This PhD thesis is structured as follow: Chapter \ref{Industrial Context and Research Framework} focuses on detailing the industrial context in which this research work is inscribed. The extrusion blow-moulding, as well as the key quality characteristics of a fuel tank are described. Subsequently, a literature review of the quality control for the extrusion blow-moulding process is presented. This would allow for the positioning of our scientific work and to subsequently define the project objectives. Chapter \ref{Machine Learning for Quality Control} describes a general method to handle the quality improvement topic using a supervised machine learning approach. The second chapter bring also a special attention to define the core concepts and approaches used in machine learning, thus serving as an introduction to machine learning algorithms and techniques extensively used in the following chapters. Chapter \ref{From Corrective to Predictive Process Control} presents a first experimental application of the method described in Chapter \ref{Machine Learning for Quality Control}. Supervised machine learning is used  to try to infer the weight of the fuel tanks given the measured process data on the machine. This chapter will also present two software-based applications developed during the presented research work. A first application makes use of an RGB camera to measure the length of the parison in real-time. The second application allows for the optimisation of some critical phases of the machine such as the start-ups and purge cycles. In Chapter \ref{Thickness inference using thermal imaging} we show how thermal imaging, or better the surface-decay temperature, can be used to infer the thickness of fuel tanks through a learning algorithm. Three data-driven pipelines are proposed to leverage machine learning and deep learning to infer the thickness value of some critical areas. Finally, in the general conclusion we resume our contributions and  we present a few research perspectives that can be addressed in the future to push forward the research on this topic.  

\clearpage