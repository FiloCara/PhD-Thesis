\chapter*{Introduction TO BE FINISHED}
\addcontentsline{toc}{chapter}{Introduction}
\thispagestyle{empty}

This thesis is motivated by the recent advances

Industry 4.0 is also a new way of looking at performance, with a more precise and immediate vision (based on real-time indicators) of the entire production chain, but also the optimisation of production through the use of data-driven methods. In the context of an interconnected plant, the large amount of data collected from different sources—production equipment and systems as well as enterprise—can be helpful in taking decision and contributes to a continuous improvement process. In particular, we think that the integration of machine learning models inside our complex industrial processes can reduce the non-quality costs with the increase of the overall equipment effectiveness (OEE). This thesis explore how data-driven methods and, in particular, Machine Learning and Deep Learning could be applied in manufacturing to improve the quality of the produced parts. The experimental part will be predominant in this research work FINIRE. 



In our opinion the overall quality improvement of the manufactured parts is a consequence of two majors FINIRE

\begin{itemize}
    \item The manufacturing process optimisation:
    \item The quality control optimisation:
\end{itemize}

Both approaches could be conducted through the use of statistical learning approaches.

\section*{Research Contribution}


\section*{Thesis structure}

This PhD. thesis is structured as follow: chapter 1 

% This PhD. thesis is structured as followed: after this introduction, chapter 1 focuses on detailing the current industrial context in manufacturing. A focus on the reasons which led us to consider methods from the field of AI to address the challenges arisen by AVI. The first chapter bring also a special attention to define the core concepts and approaches used in AI, thus serving as an introduction to AI extensively used in the other chapters. The first chapter ends by defining the objectives of this thesis and the criterions used to evaluate the approach proposed in this thesis. Chapter 2 highlights the reasons motivating the use of a standardized dataset. The formers led us to propose the DMU-Net dataset, freely available on: www.dmu-net.org, and extensively described in this chapter. Nonetheless, by being quite recently introduced, this dataset does

\clearpage