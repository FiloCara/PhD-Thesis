\chapter*{Introduction}
\markboth{\MakeUppercase{Introduction}}{\MakeUppercase{Introduction}}
\addcontentsline{toc}{chapter}{Introduction}


The development of new technologies such as machine learning (and deep learning), IoT and cloud computing are opening up new research perspectives in the manufacturing industry. Industry 4.0 holds the promise of increased flexibility in manufacturing, along with mass customisation, better quality, and improved productivity. In this context, Plastic Omnium Clean Energy System aims to leverage these new technologies in order to keep its leadership in the manufacturing industry of fuel tanks. For Plastic Omnium Clean Energy System, Industry 4.0 is a new way of looking at performance, with a more precise and immediate vision (based on real-time indicators) of the entire production chain, but also the optimisation of production through the use of data-driven methods. In the context of an interconnected plant, the large amount of data collected from different sources —production equipment and systems as well as enterprise— can be helpful in taking decisions and contributes to a continuous improvement process. In particular, Plastic Omnium Clean Energy System thinks that the integration of machine learning models inside a complex industrial process can reduce the non-quality costs with the increase of the Overall Equipment Effectiveness (OEE). This research work focuses on the quality improvement of fuel tanks produced through the extrusion blow-moulding manufacturing process. Extrusion blow-moulding takes a thin-walled tube called a \textit{parison} that has been formed by extrusion, entraps it between two halves of a larger diameter mould, and then expands it by blowing air into the tube, forcing the parison out against the mould. The fuel tank produced through this manufacturing process must respect some dimensional and geometrical constraints to comply with customer specifications. The thickness of the tank over the whole surface must be sufficient to ensure the robustness of the part and therefore its safety, while avoiding an excessive and unnecessary weight of the finished product. Unfortunately, measuring the thickness of a hollow part is a time-consuming operation that requires several minutes of work and that cannot be done online for each part. As a consequence, only a subset of the produced parts can be measured. One set of statistical tools for applying such a screening is acceptance sampling. Using such tools enables decision makers to determine what action to take on a batch of products. Decisions based on frequency testing, rather than on 100\% inspection, are more expedient and cost effective but it cannot guarantee the conformity of all parts of the population from which the sample was drawn. In order to reinforce the control of parts, the tank weight is measured for 100\% of the manufactured parts. The weight is an indicator of how much material is composing the part and allows for overall control of part quality. Unlike thickness, which has to be measured in several areas of the tank and cannot be carried out online for all parts, weight can be easily measured for all parts and can provide an overall information about the amount of material composing the fuel tank. This thesis explores how data-driven methods, and in particular machine learning and deep learning, can be applied in the industrial context of the extrusion blow-moulding in order to improve the quality of the fuel tanks produced. Supervised machine learning is used as a tool to discover hidden patterns between process parameters of the machine and quality of the parts that have been manufactured.

In our opinion, the overall quality improvement of the manufactured parts can be achieved in two ways:

\begin{itemize}
    \item through the manufacturing process optimisation;   
    \item by improving quality control through inspection of all parts, it is possible to react faster to quality non-conformities and avoid sending customers parts that do not comply to the specifications, which may cause a quality recall.  
\end{itemize}
%
We claim that machine learning, and more generally data-driven methods, can be either used to optimise the process and the quality control. By modelling the relationship between process and quality data, using a data-driven method, it is possible to infer the quality of a part given a new set of input data. Moreover, by leveraging interpretable machine learning algorithms it is possible to identify the parameters that most affect the quality of the final part.    

The experimental part is predominant in this research work. Firstly, we rely on experimentation and measurement to get all the data needed to fit the statistical models. The machine has been equipped with new sensors, such as \textit{RGB} or thermal cameras, in order to collect a new data. These new sources of data, combined with process data already available within the company, will constitute the entry point for training our machine learning models. In addition, this research work has enabled us to develop some industrial software applications that add value to the overall extrusion blow-moulding manufacturing process. The development of these applications is an outcome of  our data analysis and is one of the contributions of this research work.

\section*{Thesis structure}

This PhD thesis is structured as follow: Chapter \ref{Industrial Context and Research Framework} focuses on detailing the industrial context in which this research work takes place. Extrusion blow-moulding, as well as the key quality characteristics of a fuel tank are described. Then, a literature review of quality control for extrusion blow-moulding process is presented. This allows us to position of our work and to subsequently define the objectives of the project. Chapter \ref{Machine Learning for Quality Control} describes a general method  for quality improvement using a supervised machine learning approach. The second chapter also pays special attention to defining the core concepts and approaches used in machine learning, thus serving as an introduction to the machine learning algorithms and techniques extensively used in the following chapters. Chapter \ref{From Corrective to Predictive Process Control} presents a first experimental application of the method described in Chapter \ref{Machine Learning for Quality Control}. Supervised machine learning is used to infer the weight of fuel tanks from the process data measured on the machine. This chapter also presents two software applications developed during the course of the research work presented. A first application uses an RGB camera to measure the length of the parison in real time. The second application allows for the optimisation of some critical phases of the machine such as start-ups and purge cycles. In Chapter \ref{Thickness inference using thermal imaging} we show how thermal imaging, more precisely the measurement of surface-decay temperatures, can be used to infer the thickness of fuel tanks using a learning algorithm. Three data-driven pipelines are proposed to leverage machine learning and deep learning to infer the thickness of some critical areas. Finally, in the general conclusion we summarise our contributions and present some research perspectives that can be addressed in the future to push forward the research on this topic.  

\cleardoublepage